\documentclass{scrartcl}

\usepackage{acronym}
\usepackage{hyperref}

\usepackage[a4paper, total={15cm, 25cm}]{geometry}
\usepackage[parfill]{parskip}

\usepackage{todonotes}

\title{Understanding the Structure of District Heating Networks from Measurement Data}
\subtitle{an Exposè}
\author{Fabian Weik}

\begin{document}

    \maketitle

    \section{Acronyms}

    \begin{acronym}
        \acro{DHN}[DHN]{District Heating Network}
    \end{acronym}

    \section{Introduction}

    Recently, \ac{DHN} have been of ever growing importance.
    Especially in Germany
    \todo{source of government decisions}

    For efficient operation, we want to simulate these networks.
    One example is AD-NET or rather AD-HEAT
    \todo{source to website}

    For the simulation, we require knowledge about all pipes, producers, and consumers.
    Unfortunately, this is not guaranteed.
    Especially in older systems the structure (or topology) of the network might not be known in parts or completely.
    \todo{source}

    To enable the simulation of such networks, it would be helpful to recover the structure of the network from measurement data.
    This work aims at exploring possibilities, their requirements and limitations.
    We don't necessarily require the recovered structure to be identical to the actual structure of the real-wold network --- rather  the goal is a structure that yields good simulation results.

    Measurement data: We have access to the temperature of the flow network of the \ac{DHN}.

\end{document}
