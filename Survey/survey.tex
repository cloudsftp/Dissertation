\documentclass{scrartcl}

\usepackage[a4paper, total={15cm, 25cm}]{geometry}

\setlength{\parskip}{1em}
\setlength{\parindent}{0em}

\usepackage{amsmath}

\usepackage{todonotes}
\usepackage{acronym}

\usepackage{hyperref}
\usepackage{cleveref}

\usepackage[
  style=numeric,
]{biblatex}
\addbibresource{../references.bib}

\title{Identifying the Structure of \\ District Heating Networks}
\subtitle{A Literature Review}
\author{Fabian Weik}

\begin{document}

    \maketitle

    \section{Introduction}

    The information about the structure of \acp{DHN} is sometimes outdated, missing in part, or missing completely.
    But it is needed for the precise simulation of such systems and subsequent optimization~\cite{adnetheat}.
    Identifying the structure of \acp{DHN} is therefore of high interest.

    Unfortunately, there is no literature on the identification of the structure of \acp{DHN}.
    However the identification of the parameters of pipes in \acp{DHN} is documented well in the literature.
    So is the identification of the structure of electrical \acp{PDN}.
    This is helpful, since both \acp{DHN} and electrical \acp{PDN} follow Kirchhoffs laws~\cite{Moser2016waterelectricitysimilarity}.

    We consider approaches used to both identifying the pipe parameters in \acp{DHN} and identifying the structure of electrical \acp{PDN}.
    This article is structured as follows.
    \Cref{sec:parameter-identification} compiles the literature on the identification of pipe parameters of \acp{DHN}, while \Cref{sec:structure-identification} focuses on the identification of the structure of electrical \acp{PDN}.

    \section{Parameter Identification in \acp{DHN}}
    \label{sec:parameter-identification}

    \section{Structure Identification in \acp{PDN}}
    \label{sec:structure-identification}

    \printbibliography

    \section{Acronyms}

    \begin{acronym}
        \acro{DHN}[DHN]{District Heating Network}
        \acro{PDN}[PDN]{Power Distribution Network}
    \end{acronym}
\end{document}
