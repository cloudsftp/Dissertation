\documentclass{scrartcl}

\usepackage[a4paper, total={15cm, 25cm}]{geometry}

\setlength{\parskip}{1em}
\setlength{\parindent}{0em}

\usepackage{amsmath}

\usepackage{todonotes}
\usepackage{acronym}

\usepackage{hyperref}
\usepackage{cleveref}

\usepackage[
  style=numeric,
]{biblatex}
\addbibresource{../references.bib}

\title{Identifying the Structure of \\ District Heating Networks}
\subtitle{A Literature Review}
\author{Fabian Weik}

\begin{document}

    \maketitle

    \section{Introduction}

    The information about the structure of \acp{DHN} is sometimes outdated, missing in part, or missing completely.
    But it is needed for the precise simulation of such systems and subsequent optimization~\cite{adnetheat}.
    Identifying the structure of \acp{DHN} is therefore of high interest.

    Unfortunately, there is no literature on the identification of the structure of \acp{DHN}.
    However the identification of the parameters of pipes in \acp{DHN} is documented well in the literature.
    So is the identification of the structure of electrical \acp{PDN}.
    This is helpful, since both \acp{DHN} and electrical \acp{PDN} follow Kirchhoffs laws~\cite{Moser2016waterelectricitysimilarity}.

    In this document, we explore three different areas of research connected to the task of identifying the topology of \acp{DHN}
    First, we consider the literature on identifying the pipe parameters in \acp{DHN}.
    \Cref{sec:parameter-identification} compiles our findings regarding that area of research.
    Next, we consider the literature on identifying the topology of networks in general.
    \Cref{sec:topology-identification} compiles our findings regarding that area of research.
    And finally, in \Cref{sec:topology-identification-electrical}, we focus on the literature on identifying the topology of electrical \acp{PDN}.

    \section{Parameter Identification in \acp{DHN}}
    \label{sec:parameter-identification}

    \section{Topology Identification in Networks in General}
    \label{sec:topology-identification}

    Different application areas:
    \begin{itemize}
      \item Neural Networks
      \item Genetic Networks
      \item Financial Markets
    \end{itemize}

    Approaches:
    \begin{itemize}
      \item Explicitly from Markov Parameters
      \item Minimum Spanning Tree from Distances
      \item Adaptive Observers
      \item Lyapunov Equations
    \end{itemize}

    \section{Topology Identification in electrical \acp{PDN}}
    \label{sec:topology-identification-electrical}

    The research on the identification of topologies in electrical \acp{PDN} can be roughly divided into two types.
    One group of approaches uses graph theory and linear algebra to compute the topology of networks from observed data.
    We call this group the ``Explicit Approaches''.
    The other group of approaches is much larger and uses optimization techniques to find the best fitting network topology to the observed data.
    We call this group the ``Optimization Approaches''.

    \subsection{Explicit Approaches}

    \subsection{Optimization Approaches}

    \printbibliography

    \section{Acronyms}

    \begin{acronym}
        \acro{DHN}[DHN]{District Heating Network}
        \acro{PDN}[PDN]{Power Distribution Network}
    \end{acronym}
\end{document}
